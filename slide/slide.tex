\documentclass{beamer}


%\usetheme{Darmstadt}
\usetheme{Berkeley}
\usecolortheme{rose}
\useoutertheme{tree}
\usefonttheme[onlylarge]{structurebold}
\setbeamerfont*{frametitle}{size=\normalsize,series=\bfseries}
\setbeamertemplate{navigation symbols}{}
\setbeamertemplate{blocks}[rounded][shadow=true] 
\setcounter{tocdepth}{1}

\usepackage{minted}
\usepackage{tikz}
\usepackage{stmaryrd}
\usepackage{tabularx}
\usepackage[backend=bibtex]{biblatex}
\defbibheading{bibliography}{}
\bibliography{refs.bib}

\usepackage[slantfont,boldfont]{xeCJK}
\setCJKmainfont{"SimHei"}
\setCJKmonofont{"SimHei"}
%\setCJKmainfont{Adobe Heiti Std}
%\setCJKmonofont{Adobe Heiti Std}

\XeTeXlinebreaklocale "Hz"
\XeTeXlinebreakskip = 0pt plus 1pt
% \titleformat*{\section}{\centering\Large\bf}
% \titleformat*{\subsection}{\bf}


\title{Rise of the Machines: Cyber Grand Challenge及DEFCON 24 CTF决赛介绍}
\author{宋方睿 MaskRay}
\institute{https://maskray.me}
\date{}

\newcommand{\image}[1]{
  \begin{frame}
    \includegraphics[width=\textwidth,height=0.8\textheight,keepaspectratio]{#1}
  \end{frame}
}

\begin{document}

\begin{frame}
  \titlepage
\end{frame}

\begin{frame}
  \tableofcontents
\end{frame}

\section{喵}

{
  \usebackgroundtemplate{
    \tikz[overlay,remember picture]
    \node[opacity=0.4, at=(current page.south east),anchor=south east,inner sep=0pt] {
      \includegraphics[height=\paperheight]{img/me.jpg}};
  }

  \begin{frame}
    \begin{block}{MaskRay}
      \begin{itemize}[<+-|alert@+>]
        \item 过期的算法竞赛+超算赛棍
        \item 发霉的运维+FP爱好者
        \item 变质的四届DEFCON CTF酱油
      \end{itemize}
    \end{block}
  \end{frame}
}

\begin{frame}
  \begin{block}{两个竞赛}
    \begin{itemize}
      \item Cyber Grand Challenge (CGC) Final Event,8月4日
      \item DEFCON 24 Capture the Flag,8月5$\sim$7日
    \end{itemize}
  \end{block}
\end{frame}

\section{Capture the Flag}

\begin{frame}
  \begin{block}{Capture the Flag}
    \begin{itemize}
      \item 逆向技术, 协议分析, 网络嗅探, 密码破解, 计算机取证, 编程
      \item Codegate CTF, DEFCON CTF, Hack.lu CTF, Plaid CTF, SECCON CTF
      \item 0CTF, BCTF, XCTF
    \end{itemize}
  \end{block}
\end{frame}

\begin{frame}
  \begin{block}{形式}
    \begin{itemize}[<+-|alert@+>]
      \item jeopardy,Online Judge
      \item attack-defense
      \item CGC
    \end{itemize}
  \end{block}
\end{frame}

\image{img/a.jpg}

\begin{frame}
  \begin{block}{CGC}
    \begin{itemize}
      \item Cyber Reasoning System
      \item 寻找漏洞
      \item 修补漏洞
      \item 分析攻击
      \item 设置防火墙
      \item 利用漏洞(exploit)
      \item 1280 cores, 16TB ram, 128 TB storage
    \end{itemize}
  \end{block}
\end{frame}

\section{Cyber Grand Challenge}

\image{img/cgc-timeline.png}

\begin{frame}
  \begin{block}{CGC Qualifying Event}
    \begin{itemize}
      \item 24小时分析126个challenge binary (CB)
      \item 产生让CB崩溃的交互
      \item 修补CB,保留功能,性能也影响分数
    \end{itemize}
  \end{block}
\end{frame}

\begin{frame}
  \begin{block}{DECREE环境}
    \begin{itemize}
      \item 修改的Linux 3.13.0,32位x86
      \item ELF -> CGC(可执行文件格式)
      \item allocate(mmap), deallocate(munmap), fdwait(select), random, receive(read), terminate(exit), transmit(write)
      \item SIGPIPE Ign,SIGSEGV SIGILL SIGBUS Core,其他Term
      \item 禁用address space layout randomization,禁用non-executable stack
      \item CR4寄存器禁用performance monitoring center
      \item static linking, homebrew libc
    \end{itemize}
  \end{block}
\end{frame}

\subsection{入围队伍}

\begin{frame}
  \begin{block}{CodeJitsu}
    \begin{itemize}
      \item University of California, Berkeley
      \item \href{http://bitblaze.cs.berkeley.edu/}{BitBlaze} Binary Analysis Platform: Vine(static analysis), TEMU(dynamic analysis), Rudder(symbolic execution)
    \end{itemize}
  \end{block}
\end{frame}

\image{img/codejitsu.jpg}

\begin{frame}
  \begin{block}{ForAllSecure}
    \begin{itemize}
      \item CMU教授David Brumley發起的startup,成员多来自CyLab
      \item \href{https://github.com/BinaryAnalysisPlatform}{Binary
          Analysis Platform}
      \item Plaid Parliament of Pwning是其undergraduate computer security research group。
    \end{itemize}
  \end{block}
\end{frame}

\image{img/forallsecure.jpg}

\begin{frame}
  \begin{block}{TECHx}
    \begin{itemize}
      \item GrammaTech \& University of Virginia Technology
      \item Preventing Exploits of Software of Unknown Provenance
    \end{itemize}
  \end{block}
\end{frame}

\image{img/techx.jpg}

\begin{frame}
  \begin{block}{CSDS}
    \begin{itemize}
      \item University of Idaho
      \item Jim Alves-Foss, Jia Song
    \end{itemize}
  \end{block}
\end{frame}

\image{img/csds.jpg}

\begin{frame}
  \begin{block}{DeepRed}
    \begin{itemize}
      \item Raytheon
    \end{itemize}
  \end{block}
\end{frame}

\image{img/deepred.jpg}

\begin{frame}
  \begin{block}{disekt}
    \begin{itemize}
      \item University Of Georgia
      \item 2009年成立disekt CTF战队
    \end{itemize}
  \end{block}
\end{frame}

\image{img/disekt.jpg}

\begin{frame}
  \begin{block}{Shellphish}
    \begin{itemize}
      \item University of California, Santa Barbara
      \item \href{http://angr.io}{angr}, a python framework for analyzing binaries. It focuses on both static and dynamic symbolic ("concolic") analysis
    \end{itemize}
  \end{block}
\end{frame}

\image{img/shellphish.jpg}

\image{img/cgc-cfe.jpg}

\begin{frame}
  \begin{block}{CGC Final Event}
    \begin{itemize}
      \item 96轮
      \item 比赛开始时CRS接收CB,每个CB以类似\texttt{socat tcp-l:9999 exec:cb}的形式提供服务
      \item 每轮为每个$(round,team,service)$产生分数,$(*,team,*)$和为该队伍累计分数
    \end{itemize}
  \end{block}
\end{frame}

\image{img/cgc-rack.png}

\begin{frame}
  \begin{block}{$(round,team,service)$}
    \begin{itemize}
      \item $score = 100\times availability\times security\times evaluation$
      \item $availability \in [0,1]$,通过poller的比例和内存时间开销
      \item $security \in \{1,2\}$,被其他CRS攻击成功?
      \item $evaluation \in [1,2]$,攻击其他CRS
    \end{itemize}
  \end{block}
\end{frame}

\begin{frame}
  \small
  \begin{table}
    \begin{tabularx}{\textwidth}{|l|X|X|}
      \hline
      & attack-defense & CGC \\\hline
      题目数量 & ~6 & 92 challenge sets(CFE)/8(DEFCON CTF) \\\hline
      流量 & 主办方提供tcpdump & 自行在1999/udp接收(服务编号,连接号,流序号,消息长度等) \\\hline
      平台 & amd64, aarch64, mipsel, $\ldots$ & DECREE \\\hline
      服务 & 可ssh,替换服务文件 & API提交修补过的 \\\hline
      可用性检测 & 主办方伪装成其他队伍检测 & 平台测试提交的CB \\\hline
      攻击方式 & 手工, 程序 & 提交proof-of-vulnerability \\\hline
      flag & 主办方每轮生成,服务程序有权限读取的文件 & magic page填充随机值 \\
      防火墙 & executable wrapper & 类snort规则 \\\hline
    \end{tabularx}
  \end{table}
\end{frame}

\section{CB, Poller, POV, IDS}

\begin{frame}
  \begin{block}{Challenge binary}
    \begin{itemize}
      \item 题目用的可执行文件,特意设置了若干漏洞
      \item 分析、修补、利用
      \item API上传修补后的CB
    \end{itemize}
  \end{block}
\end{frame}

\begin{frame}
  \begin{block}{Poller generator}
    \begin{itemize}
      \item 检测CB可用性
      \item finite state automaton
      \item 至少1000000不同输出
    \end{itemize}
  \end{block}
\end{frame}

\begin{frame}[fragile]{}
  \tiny
  \begin{center}
    \begin{minipage}{0.5\textwidth}
      \begin{verbatim}
nodes:
- name: start
- name: top
- name: endIt
- name: printAirports
- name: addAirport
- name: deleteAirport
- name: findRoutes

edges:
- start: top
- top: printAirports
- printAirports: top
- top: addAirport
- addAirport: top
- top: deleteAirport
- deleteAirport: top
- top: findRoutes
- findRoutes: top
- top: endIt
  weight: .20
      \end{verbatim}
    \end{minipage}
  \end{center}
\end{frame}

\begin{frame}
  \begin{block}{Proof of vulnerability}
    \begin{itemize}
      \item C编写的CGC可执行文件
      \item 构建方式和CB相同
      \item Type 1 \& Type 2
    \end{itemize}
  \end{block}
\end{frame}

\begin{frame}
  \begin{block}{Type 1 vulnerability}
    \begin{itemize}[<+-|alert@+>]
      \item 控制EIP与8个general purpose register中任意一个
      \item 如果证明能控制?
      \item Challenge response, POV程序向平台宣称能控制寄存器的特定20 bits,平台指定20 bits的值
      \item 程序崩溃时两个寄存器的值与challenge匹配
    \end{itemize}
  \end{block}
\end{frame}

\begin{frame}
  \begin{block}{Type 2 vulnerability}
    \begin{itemize}
      \item magic page
      \item CGC可执行文件执行时,0x4347c000处内核分配一页,填充随机值
      \item Challenge response,平台指定要输出magic page指定区间内的4字节
      \item POV程序设法获取
    \end{itemize}
  \end{block}
\end{frame}

\begin{frame}
  \begin{block}{Intrusion detection system (IDS)}
    \begin{itemize}
      \item 防火墙规则
      \item 可以阻挡攻击,也可能误伤poller generator
      \item domain-specific language
    \end{itemize}
  \end{block}
\end{frame}

\begin{frame}
  \begin{block}{Proof of vulnerability (POV)}
    \begin{itemize}
      \item C编写的CGC可执行文件
      \item 构建方式和CB相同
      \item Type 1 \& Type 2
    \end{itemize}
  \end{block}
\end{frame}

\begin{frame}
  \begin{block}{Oracle}
    \begin{itemize}
      \item Input: CB, POV, IDS
      \item Output: score, packet captures, others' CB \& IDS
      \item 可以下载其他队伍的CB和IDS
    \end{itemize}
  \end{block}
\end{frame}

\section{Shellphish的CRS}

\begin{frame}
  \begin{block}{实现}
  \end{block}
\end{frame}

\begin{frame}
  \begin{block}{统计}
    \begin{itemize}
      \item 82 Challenge Sets
      \item 2442 exploits generated
      \item longest exploit: 3791 lines of C code
      \item shortest exploit: 226 lines of C code
    \end{itemize}
  \end{block}
\end{frame}

\section{DEFCON 24 CTF Finals}

\begin{frame}
  \begin{block}{b1o0p}
    \begin{itemize}
      \item \textcolor{blue}{bl}ue-lotus + \textcolor{red}{0op}s = \textcolor{blue}{b1}\textcolor{red}{o0p}
      \item blue-lotus成立于清华大学网络与信息安全实验室,是中国首支入围DEFCON CTF全球决赛的战队
      \item 上海交通大学0ops成立于2013年,成员主要来自于计算机系密码学与计算机安全实验室、信息安全工程学院等,大陆首支国际CTF赛事冠军战队,2015年ctftime排名第3。
    \end{itemize}
  \end{block}
\end{frame}

\image{img/b1o0p.jpg}

\section{DEFCON 24 CTF CB}

\begin{frame}
  \begin{block}{b1o0p}
    喵
  \end{block}
\end{frame}

\section{References}

\begin{frame}[t,allowframebreaks]
  \begin{itemize}
    \item http://kb.hitcon.org/post/131158681227/cyber-grand-challenge-%E7%B0%A1%E4%BB%8B
  \end{itemize}
  \printbibliography
 \end{frame}

\end{document}
